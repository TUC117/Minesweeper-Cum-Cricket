% LaTeX Template for short student reports.
% Citations should be in bibtex format and go in references.bib
\documentclass[a4paper, 11pt]{article}
\usepackage[top=3cm, bottom=3cm, left=2cm, right=2cm]{geometry}
\usepackage[utf8]{inputenc}
\usepackage{textcomp}
\usepackage{graphicx}
\usepackage{amsmath, amssymb}
\usepackage{bm}
\usepackage[pdftex, bookmarks, colorlinks, breaklinks]{hyperref}
\usepackage{memhfixc}
\usepackage{pdfsync}
\usepackage{fancyhdr}
\usepackage{subcaption}
\usepackage{listings}
\usepackage{xcolor}
\usepackage{verbatim}
\pagestyle{fancy}
\fancyhf{}
\rhead{CS 104 Project}
\lhead{Trasula Umesh Karthikeya}
\rfoot{\thepage}

\title{CS 104 Project}
\author{Trasula Umesh Karthikeya}

\lstset{
    language=JavaScript,
    basicstyle=\ttfamily,
    keywordstyle=\color{blue},
    commentstyle=\color{green!60!black},
    stringstyle=\color{red},
    numbers=left,
    numberstyle=\tiny,
    breaklines=true,
    showstringspaces=false
}

\begin{document}
\maketitle
\tableofcontents
\pagebreak

\section{Introduction}
My project was about creating an online Cricket Cum Minesweeper game\ldots
I developed a few HTML pages to make this possible\ldots
The main results were \ldots

\section{Background}
I went through some online code to understand how a Minesweeper game works, CSS styles, JS code format, etc. \\
I made modifications to some of the codes and observed the changes in the output. I also used ChatGPT to debug errors and understand parts of other codes that I found challenging.

\section{Game Design}
\subsection{Main Page}
The main page looks like Figure \ref{fig:main-page}.
\begin{figure}[h!]
    \centering
    \includegraphics[scale=0.35]{main.png}
    \caption{Main Page}
    \label{fig:main-page}
\end{figure}

It contains:
\begin{itemize}
    \item A box to input the size of the grid.
    \item Checkbox 1: Singleplayer
    \item Checkbox 2: Multiplayer
    \item Start Game Button
\end{itemize}

\subsection{Singleplayer Page}
The singleplayer page looks like Figure \ref{fig:singleplayer-page}.
\begin{figure}[h!]
    \centering
    \includegraphics[scale=0.35]{Singleplayer.png}
    \caption{Singleplayer Page}
    \label{fig:singleplayer-page}
\end{figure}

In the singleplayer game:
\begin{itemize}
    \item 11 fielders are arranged hidden.
    \item Clicking on the grids reveals your score on that grid. You can continue the game until you click on a grid with a fielder.
    \item You may receive power-ups that double your current score if you are lucky.
\end{itemize}

\subsection{Multiplayer Page}
The multiplayer game consists of two pages shown in Figure \ref{fig:multiplayer-pages}.

\begin{figure}[h!]
    \centering
    \begin{subfigure}[b]{0.5\textwidth}
        \includegraphics[width=\textwidth]{mul1.png}
        \caption{Multiplayer Page 1}
    \end{subfigure}
    \hfill
    \begin{subfigure}[b]{0.4\textwidth}
        \includegraphics[width=\textwidth]{MUL 2.png}
        \caption{Multiplayer Page 2}
    \end{subfigure}
    \hfill
    \begin{subfigure}[b]{0.4\textwidth}
        \includegraphics[width=\textwidth]{MUL3.png}
        \caption{Multiplayer Page 3}
    \end{subfigure}
    \caption{Multiplayer Pages}
    \label{fig:multiplayer-pages}
\end{figure}
\pagebreak
\subsection{OUTs and Powerups}
\begin{figure}[h!]
    \centering
    \begin{subfigure}[b]{0.5\textwidth}
        \includegraphics[width=\textwidth]{out.png}
        \caption{OUT}
    \end{subfigure}
    \hfill
    \begin{subfigure}[b]{0.4\textwidth}
        \includegraphics[width=\textwidth]{Powerups.png}
        \caption{PowerUP}
    \end{subfigure}
        \caption{Multiplayer Pages}
    \label{fig:OUTs and Powerups}
\end{figure}

\section{Implementation}
\subsection{Single player}
    In the single player game I have created a grid and arranged 11 players randomly in \textit{n*n} using \textit{Math.random()} function.\\ \\
    \textbf{I have used the following functions:} 
    \begin{itemize}
        \item \textbf{handleCellClick()} \\ Using this function we can handle any grid using a mouse click. It has two \textit{3 if-else} conditions. \\ One for \textit{if clicked grid has a mine} (Game will be ended by calling \textit{endGame()} function). \\ Second one checks weather game is ended or not using a bool variable \textit{gameEnded}. \\ Third one is for the powerup. I have created a random number less than \textit{n*n} and divided it by 10 and then compared the remainder with 7. If this is true then you will get a powerup. I have done this operation to as a random operation as we need to assign a random grid for power up.\\And score is also generated in this function score can be from 0 to 6 except 5. I have exculded 5 by using a while function \textit{"while (randomnumber == 5){randomnumber = Math.floor(Math.random() * 7);"}}\\
        And there will be variable called \textit{Score} which calculate the final score by adding the individual scores of each grid and also Doubling when it got a powerup. Individual scores of each grid will be visible on that particular grid by using \textit{cell.textContent = randomnumber;} here \textit{randomnumber} is the present score for the ball.
        \item \textbf{endGame()}
        \\ It will assign the \textit{bool gameEnded} as true. It will clear the grid by using \\ \textit{    let cleanet = document.getElementsByClassName("table-container"); \\cleanet[0].style.display = "none";}\\
        I have displayed the umpire gif by using \\ \textit{let out = document.getElementById("fig");\\out.style.display = "block";}
        \\ It will have an if cond. containing the variable \textit{endg} (Used as a short cut for endgame) it is number of times you have clicked the grid if \textit{"(n*n - endg)"} is equal to 11 then the player who played the will be declared as the winner in this \textit{Singleplayer Game}. 
        \item \textbf{Restart()}
        \\ This function will be actived by clicking \textbf{Play Again} button in Singlplayer page. \\ This function reset the game by using \textit{location.reload();} which reloads the current web page.
        \item For more information refer \cite{latex3}, \cite{latex4}

    \end{itemize}
    \subsection{Multi-Player}
    For the implementation of Multiplayer game I have used two html pages \textit{Multiplayer.html} and \textit{supmul.html} (short name as support for Multiplayer).\\\\ \textbf{Multiplayer.html} have the following features:\\
    \begin{itemize}
        \item  This page will have two arrays which will be used later one for player names and other for player scores (all the scores were assigned to zero).
        \item  This page will have a box that takes number of players (greater than 1) as input and then their names and stores them in the array playernames.
        \item Now this page will store: \\ 
        
        \begin{tabular}{|c|c|}
        \hline
        \textbf{Variable} & \textbf{How it is stored in Browser's local storage} \\
        \hline
        \textit{playerscores} & localStorage.setItem("playerscores", JSON.stringify(playerscores)); \\ \hline
        \textit{playernames} & localStorage.setItem("playernames", JSON.stringify(playernames));\\ \hline
        \textit{number} & localStorage.setItem("number", number); \\ \hline
        \textit{numplayers} & localStorage.setItem("numplayers", numplayers);\\ \hline
        \textit{numtimes} & localStorage.setItem("numtimes", numtimes);\\
        \hline 
        \end{tabular}
    \end{itemize}
\pagebreak


   \textbf{supmul.html} will work based on the following functions and strategies:
    \begin{itemize}
        \item First we will use the following to obatin the respective variable:\\
    \begin{tabular}{|c|c|}
        \hline
        \textbf{Variable} & \textbf{How it is stored in Browser's local storage} \\
        \hline
        \textit{playerscores} & JSON.parse(localStorage.getItem("playerscores")); \\ \hline
        \textit{playernames} & JSON.parse(localStorage.getItem("playernames"));\\ \hline
        \textit{number} & localStorage.getItem("number"); \\ \hline
        \textit{numplayers} & localStorage.getItem("numplayers");\\ \hline
        \textit{numtimes} & localStorage.getItem("numtimes");\\
        \hline 
        \end{tabular}
        \item My Strategy in this multiplayer is :
          \begin{itemize}
              \item Calling this page recursively by number of players times
               \item Each time when the page is called it updates the array \textit{playerscores} by pusing the score of the each player by using \textit{playerscores.push(score);}.
                \item Then when then final player clicks the continue button the grid and some extra elements will be disappeard from the screen. And scoreboard will be visualised.
                \item \textit{Scoreboard} will show the players names and their respective scores. It will also show the winner name \textit{Player with maximum score}. 
                \item If any of the two players have same score then it will show \textit{There is a tie } and the player names.
          \end{itemize}
        \item Function used in this page are :
            \begin{itemize}
                \item \textit{handleCellClick(event)}
                \item \textit{endGame()}
                \item \textit{showscores()}
                \item \textit{Restart()}
                \item \textit{createagain()}
            \end{itemize}
        \item function \textit{handleCellClick(event);}
        \\ This is the same function that we have used in the case of \textbf{Single-Player}.
        \item funtion \textit{endgame();}
        \begin{lstlisting}
            gameEnded = true;
            let cleanet = document.getElementsByClassName("table-container");
            cleanet[0].style.display = "none";
            let out = document.getElementById("fig");
            out.style.display = "block";
            playerscores.push(score);
            numtimes++;
        \end{lstlisting}
         The function is same as that in Single-Player but only difference is it also counts numtimes the game played by increasing the variable \textit{numtimes} by one.
        \item function \textit{Restart()} 
           \\ This will have a slighter difference with the function in Single Player.

            \begin{lstlisting}
                if (numtimes == numplayers) {
                showscores();
            }
            else {
                localStorage.setItem("playerscores", JSON.stringify(playerscores));
                localStorage.setItem("playernames", JSON.stringify(playernames));
                localStorage.setItem("number", number);
                localStorage.setItem("numplayers", numplayers);
                localStorage.setItem("numtimes", numtimes);
                window.location.href = "supmul.html";
            }
        }
            \end{lstlisting}
            This will show the \textbf{Score board} by comparing numtimes with numplayers or else it will update the array \textit{playerscores}. 
          
        \item function \textit{showscores()}
          \begin{itemize}
              \item Intially this will remove the \textit{grid, type-writer, buttons, etc.} and \textit{Scoreboard, Playernames, Playerscores, winner}
              \item We will find the index of maximum score by 
              \begin{lstlisting}
                  let maxnumber = -Infinity;
            let maindex = -1;
            for (let i = 0; i < playerscores.length; i++) {
                if (playerscores[i] > maxnumber) {
                    maxnumber = playerscores[i];
                    maindex = i;
                }
            }
              \end{lstlisting}
              \item Now if any two player score the same maximum score then a tie will be given to them or else the player who scores maximum will be declared as winner.
              \begin{lstlisting}
                              let j = 0;
            let equal = [];
            equal.push(maindex);
            for (j; j < playerscores.length; j++) {
                if (playerscores[maindex] == playerscores[j] && j != maindex) {
                    equal.push(j);
                }
            }
            if (equal.length == 1) {
                let winner = playernames[maindex];
                document.getElementById("output3").innerHTML = "The Winner in the game is " + winner;
            }
            else {
                let ties = [];
                for (let z = 0; z < equal.length; z++) {
                    ties.push(playernames[equal[z]]);
                }
                document.getElementById("output3").innerHTML = "There is tie between " + ties;
            }
        }
              \end{lstlisting}
              \item In the above code I have declared a variable j, array equal. 
              \item Then it will check score of the player with \textit{maindex} with other indices and if there any the index of that score will be noted in array \textit{equal}. And name of the players will be noted in array \textit{ties}.
              
          \end{itemize}
        \item Function \textit{createagain()}
        \\ This will create the grid again and this function will be activated when clicked on \textit{Continue} button.\\ The code is as follows:
        \begin{lstlisting}
                        for (let i = 0; i < numRows; i++) {
                let row = document.createElement('tr');
                grid.appendChild(row);

                for (let j = 0; j < numCols; j++) {
                    let cell = document.createElement('td');
                    cell.dataset.row = i;
                    cell.dataset.col = j;
                    cell.addEventListener('click', handleCellClick);
                    row.appendChild(cell);
                }
            }

            // Randomly place Fielders on the grid
            for (let i = 0; i < numfielder; i++) {
                let randomRow = Math.floor(Math.random() * numRows);
                let randomCol = Math.floor(Math.random() * numCols);
                let cell = grid.rows[randomRow].cells[randomCol];
                cell.dataset.hasMine = 'true';
            }
        \end{lstlisting}
        \item For more information refer \cite{latex2}

    \end{itemize}
    I have used \cite{latex1} for debugging.
    \pagebreak
\bibliographystyle{plain}
\bibliography{report}
\end{document}
